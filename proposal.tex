% Please keep it to 80 columns, no tabs, no trailing whitespace
% and no Windows line endings (\r\n)
\documentclass[10pt,a4paper,oneside]{report}
\usepackage[utf8]{inputenc}
\usepackage{graphicx}
\usepackage{lscape}
\usepackage{float}
\usepackage{pdfpages}
\usepackage{graphics}
%% \usepackage{tabularx}
\usepackage{tabularx}
\begin{document}
\title{Group 8 Integrated Project Proposal\\Project Name Here}

% Names in alphabetical order.
% The mark is for the proposal part.
\author{
  Chen Guangyu (20\%)\\
  Kowalczyk Mateusz (20\%)\\
  Luff Katie (20\%)\\
  Sampson Robert (20\%)\\
  Singh Aman (20\%)\\ }
%\date{}
\maketitle
\section*{Background}
During this integrated project, we are planning to develop a GPS application for Android phones. There will be two main purposes of our system, the first being a navigation system for new students and any visitors the university may have. It will use the GPS coordinates of the users phone to map where they are on campus and plan the shortest route to their destination. The other purpose is to provide students with information of where there are computers available on campus, and link this to the GPS so the user knows the nearest available computer to them.
We believe there is a gap in the market for this, as there is currently no route planning system around campus, and many students get frustrated around exam periods trying to find a free computer to complete their work. The idea of our proposed system being a phone application means that it can be easily accessed by students, as many have smart phones. We believe that this application will help students settle into university life quicker, as well as helping to take away any unnecessary stress around exam periods.

\section*{Research}

\subsubsection*{Google Maps \& Bing}
For our domain research we looked into Google Maps as its key feature is also a path finder. Some good features that we are going to consider in our project is that is has multiple paths that you can choose from between two points. This is good as the maps don't take into account how busy the routes are so a longer distance may take less time. Another good aspect of the system is that you can have multiple destinations, and can save your favourite places. This would be useful in our system as lectures tend to be in the same building so you don't have to type in the rooms each time you want to go there. Another feature is that it includes reviews of places of interest which could be beneficial to our system.

We also compared Google Maps with Bing Maps and compared features. Most of the core features are the same, yet Google Maps has, in our opinion, a better user interface and more features. Bing Maps doesn't have the street view feature which is one that we are considering implementing at this point of time.

We are going to take some of the features from Google Maps into account while planning our project, although as Google Maps is developed xy large corporation, so it doesn't know features of the area which we will know, such as shortcuts around campus which we would like to include in our system.


\begin{figure}[H]
 \centering
 \includegraphics[keepaspectratio, width=\textwidth]{googlemapsexample.png}
 \caption{Google Maps}
\end{figure}

\section*{Users \& significance to those users}
The tool we will be developing will be primarily aimed towards the students of Bath University.  The GPS navigation system will primarily be aimed at new students who do not know their way around the university. And the computer finder will primarily be aimed at students who are not living on campus as they will not be able to easily access a computer whilst on campus. Many of our key features are designed to aid the process of learning by reducing time wasted trying to find a suitable work station on campus. Our focus on the specific user group of students does not limit the use of our system, as the university has many industrial visitors who may not know their way around campus.

We have gathered the incentive to develop our system as all members of the team have been in the position of being lost on campus, and thought that the use of a route planner would be beneficial. In addition to this, as second year students now living off campus, we are finding it increasingly difficult to find computers to do work and would benefit from a system similar to the one we are proposing.  Although there is a web page that tells you if there are computers available, it does not inform you if the room currently in use and it is a laborious process to try and access the information relevant. We are striving to develop an application combines all the key aspects of systems, whilst being simple and easy to use so that the user only needs to use our application to find the relevant information.

\clearpage
\section*{Programme and Methodology}
\subsection*{Overall aims of the project}
After deciding on an idea for a project, the team laid out some goals which we would like to achieve, which are separate from the actual implementation. These should enable us to work effectively in a team, and to demonstrate our skills in software engineering. We can also use these goals at the end of the project to assess our performance, and analyse how well the project has gone.

At the highest level, we aim to create a working application which informs students of the location of free computers around campus, and uses GPS navigation to direct the student to the closest free computer. We aim to do this through the effective use of our design model, as well as user involvement through requirements gathering and also during testing.

Another aim is to improve on the existing systems available to students. We wish to create something that is both more usable and more functional than anything that exists in the same domain. To achieve this, we must first carry out research on the current systems, and interview potential users, to find out what areas we can improve in. More generally, we must find out more about usability and also what are the main components a system needs to make it more usable.
\subsection*{Other features of the application}
\begin{description}
\item[Free Computers] \hfill \\
This feature will provide the users with the real time information about free computers available on campus so that they don't have to look around  for computers. For this we needed real time information regarding the availability of computers at any current moment. After doing some research we found out that this information can be obtained from BUCS services website: http://www.bath.ac.uk/bucs/services/pacs/where.html
Our goal is to extract relevant information regarding free computers from this website, save the information in a data structure and apply algorithms to find the nearest computer room from user's current position.
\item[Weather widget] \hfill \\
This feature will provide the users with the real time information about the weather on campus. Lectures and Examinations are re-scheduled and sometimes cancelled due to bad weather conditions. Such a feature will be very helpful for the students since they can check the weather status directly from their smart phones. For this feature we need real time information regarding the weather at any current moment. Our goal is to  present this information using the Android weather widget. One additional feature we are adding to this is the background image of the application will change automatically depending on the weather conditions.
\item[Restaurant, Café, Shop finder] \hfill \\
This feature will provide the users with information regarding various shops and eateries on campus. A user can simply click an image representing food or a café and the application will guide the user to the nearest location.
\end{description}

\begin{table}[H]
\hspace*{-2cm}
\newcolumntype{S}{>{\small}X}
\newcolumntype{E}{>{\small}c}
\begin{tabularx}{1.3\textwidth}{ |S|E|S|E| }
  \hline
  \bf{Objectives and milestones} & \bf{Date due} & \bf{Aim} & \bf{Dependencies} \\ \hline
  Understand the layout of campus & ~ &
  Design and create and create an interactive map application & ~ \\ \hline
  Create an architectural design model of the campus. & ~ &
  To help us think about which data structures to use & ~ \\ \hline
  Milestone 1 & ~ & ~ & NONE \\ \hline
  Research data structures within android that can be used to represent elements in the map. & ~ & Find the most efficient way to create the map. & ~ \\ \hline
  Search for efficient shortest path algorithms to be implemented on the data-structure. & ~ & The user should reach his location as fast as possible. & ~ \\ \hline
  Milestone 2 & ~ & ~ &Milestone 1 \\ \hline
  Design the graphical user interface using android GUI widgets. & ~ & App should be attractive and fun to use. & \\ \hline
  Use androids inbuilt event handling techniques to make the UI functional. & ~ & All features within the app should be functional, and the system should be robust. & ~ \\ \hline
  Milestone 3 & ~ & ~ & Milestone 2 \\ \hline
  Find a way to dynamically extract information (free computers/rooms) from a webpage. & ~ & App should provide real time information about availability of work-stations. & ~\\ \hline
  Find efficient data structures that will store this information. & ~ & Processing should be efficient and robust. & ~ \\ \hline
  Apply shortest path algorithms to each element in the data structure, and then sort the resulting list. & ~ & To find the closest computer from the users location. & ~\\ \hline
  Milestone 4 & ~ & ~ & Milestone 1 and 2 \\ \hline
  Use the android weather widget to dynamically update the weather. & ~ & To give the app user most recent and most accurate information regarding the weather. & ~ \\ \hline
  Implement a function that will change the background of the application depending on the weather conditions. & ~ & To make the app more interesting and fun to use. & ~ \\ \hline
  Milestone 5 & ~ & ~ & Milestone 1 and 2 \\ \hline
  Save important information regarding all landmarks. (Pictures, opening and closing times etc...) & ~ & To enhance the knowledge of the user. & ~ \\ \hline
  Milestone 6 & ~ & ~ & Milestone 1 and 2 \\ \hline
  Project Complete & ~ & ~ & ~ \\ \hline
\end{tabularx}
\caption{Individual measurable objectives}
\end{table}

\section*{System and programming platforms}
The software platform we are using is Android.  The reason for this is:
\begin{enumerate}
\item{Smartphones are becoming popularised among mobile users, and the largest proportion of them are using Android-based phones.}
\item{Android is open source. Which means it is easier for developers to customize software components and the software can be used on multiple devices.}
\item{Development for Android is easier for developers familiar with Java and all of the team members have learned Java to some extent.}
\item{Compared to iOS and windows phone we don't need to worry about licencing issues and this makes it easier to deliver our application to everyone without cost.}
\end{enumerate}
Android in terms of the objectives of our system
\begin{enumerate}
\item{Android provides a Map API that will make it easier to code and think about system elements such as buildings and shops as objects in the real world.}
\item{Android has a lot of documentation about how to design and manipulate graphical elements. This will make it easier to code the higher level graphical model.}
\item{Android provides a host of efficient data structures which are in-built and do not have to be coded from scratch. (Example: List View, List Adapter, Stacks, Trees etc.)}
\end{enumerate}

\begin{figure}[H]
 \centering
 \includegraphics[keepaspectratio, width=\textwidth]{androiddev.png}
 \caption{Android development environment}
\end{figure}

\section*{Work plan}
During the project planning stage, each of our group members were delegated tasks and dealt with problems that came out during the development process.
This is how we approached our project planning:
\begin{enumerate}
\item{We created a Gantt chart allowing us to have an overview of the project process.}
\item{We created a GitHub version control repository to store the files and code for our project.}
\item{We formed a Facebook group that allows us to convey our opinions and ideas to the project instantly.}
\end{enumerate}
In order to keep track of the project, we did the following:
\begin{enumerate}
\item{Our group decided to meet at least once a week to discuss the progress we have made so far.}
\item{We recorded the start and end time of each of our group meetings and documented the entire meeting.}
\end{enumerate}

\begin{figure}[H]
 \centering
 \includegraphics[keepaspectratio, scale=0.5]{meeting.png}
 \caption{Meeting minutes}
\end{figure}

\subsubsection*{Milestones for the project}
\begin{enumerate}
\item{Proposal stage}
\item{Design stage}
\item{Implementation stage}
\item{Test stage}
\item{Evaluation stage}
\end{enumerate}

% We put the generated chart on the last page.
\includepdf[pages={1}]{ganttchart.pdf}

\end{document}
